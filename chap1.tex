\chapter{Introduction}
\label{chap:intro}
\title{introduction}

\section{Background of the Study}
\cite{topo_id_mod_schultz} Chemical graph theory is an interdisciplinary science that uses graph theory to study molecular structures and correlates it to different activities, properties or phenomena. It appears that graph theory reaches out its hands to different disciplines of science and continues to grow. On one account,\cite{hararygt} it started when Cayley discovered an important class of graphs called trees in the natural setting of organic chemistry. He, then restated the problem abstractly not realizing that he already approaches to the norms of graph theory. \medskip

Moving on, this study associates us to a specific type of topological index. \cite{dbti}  According to IUPAC definition (IUPAC-International Union for Pure and Applied Chemistry), \textbf{topological index} or molecular structure descriptor is a numerical value associated with chemical constitution for correlation of chemical structure with various physical properties, chemical reactivity or biological activity.  \medskip

The topological indices of molecular graphs are widely used for establishing correlations between the structure of a molecular compound and its physico-chemical properties or biological activity. However, there are advantages as well as shortcomings. \medskip

For example, \cite{rmti} the molecular topological index has been introduced by Schultz in 1989 for characterization of alkanes by integer. This term \textbf{Schultz index} has also been frequently use for MTI. More often, topological index are used to study the different properties and behaviors of organic compounds such as hydrocarbons. Relevent issues come along, since most physico-chemical meaning of topological indices are implicit. Right now the proper way to handle it is to use a suitable topological index in constructing a specific model. \medskip

\section{Statement of the Problem}
Because topological indices as molecular descriptors continue to grow, the study focuses on the establishment of the Molecular Topological Index (MTI) of a Monocyclic Graph $C_k(s,t)$, where $C_k$ is the cycle of the graph and $s$ and $t$ are the number of pendant vertices attached to the cycle. In the study, the derivation of the MTI of graph would be from its Degree distance index and Zagreb index, $MTI(G)=DD(G)+Zg(G)$.
 
\section{Objectives of the Study}
The main objective of the study is to derive the Molecular Topological Index $MTI$ of the Monocyclic graph $C_k(s,t)$, such that $k,s,t$ are all non-negative integers and $k> 3$. The study will commence by 

\begin{enumerate}
\item generalizing the Wiener index of the graph $C_k(s,t)$
\item derive the Degree distance index of the graph $C_k(s,t)$, using the results above
\item generalize the Zagreb index of the graph $C_k(s,t)$
\item use the Degree distance index and the Zagreb index to calculate the MTI of the monocyclic graph $C_k(s,t)$
\end{enumerate}

The study would verify the results using manual inspection and calculation, using applications (Microsoft Excel) and simple codes written in C (Codeblocks).

\section{Scope and Delimitation of the Study}
The study is limited only to undirected, connected Monocylic graphs with two adjacent vertices at $C_k$ with attached $s$ and $t$ pendant vertices respectively. The study would only concern to undirected graphs, that would mean the absence of multiple loops and edges.

\section{Significance of the Study}
The study would be a good outset for Chemical graph theory, especially that there are organic compounds that are monocyclic by nature. Results would be useful as molecular descriptors that would be helpful for physical, biological and chemical activity of the compound being studied. Furthermore, this would be a good start-up for future studies especially for those field of interest includes MTI and Monocyclic graphs.

\section{Review of Related Literature}
\cite{eff_enum} Monocyclic graph is an undirected connected graph
containing exactly one cycle. This is also called a unicyclic graph. Suppose we take phenol as an example. \cite{monocyclic_ex} For the last 10 years, it has been studied that there are at least 30 hydroxy- and polyhydroxybenzoic acids reported to have biological activities. According to a source, this has been viewed to benefit the human species in lieu of its distribution, ecological and biological importance, that would also lead to the development of new pharmaceutical and agricultural products. To deepen the study of the behavior of a chemical compound, researchers adhere concern to its molecular structure. A number of molecular descriptors had been used in the field including physical-chemical parameters, 3D descriptors and etc. \medskip

\cite{jds-172} Many scientists prefer the use of topological indices as a molecular descriptor especially in evaluating compound toxicity and predicting biological activity. There are tons of topological indices available, however the main concern of this study is the Schult'z Molecular Topological Index of a Monocyclic graph $C_k(s,t)$. \medskip
 
The most use topological index recorded is the Wiener index. There are known results for this type topological index that are already recorded and had been used in the field. Basically, the known results would be a great use in completion of this study. 